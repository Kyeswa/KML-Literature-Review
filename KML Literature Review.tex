\documentclass[12pt,a4paper,final]{KML Lit review}
\usepackage[utf8]{inputenc}
\usepackage{amsmath}
\usepackage{amsfonts}
\usepackage{amssymb} \usepackage[english]{babel}
\usepackage{hyperref} \hypersetup{colorlinks=true, linkcolor=blue, filecolor=magenta, urlcolor=cyan, }   \urlstyle{same} 


\begin{document}

%\selectlanguage{english} %%% remove comment delimiter ('%') and select language if require

\noindent \textbf{A LITERATURE REVIEW ON THE KEYHOLE MARKUP LANGUAGE.,}

\begin{abstract}
This paper is written to satisfy a literature review on the KML one of google's prised development applications. Sources of the literature have been mainly through internate searches and are represented throughout the whole document. I believe that this review will give a deeper incite about the subject matter and a contribution to available research.
\end{abstract}

\noindent \textbf{Introduction }
\par
Keyhole Markup Language(KML) is an XML notation for expressing geographic annotation and visualization within Internet-based, two-dimensional maps and three-dimensional Earth browsers. KML was developed for use with Google Earth, which was originally named Keyhole Earth Viewer. It was created by Keyhole, Inc, which was acquired by Google in 2004[1].

\noindent \textbf{Claims}
\par
Some sources claim that if you understand HTML/XML you will have little problem understanding the syntax of KML[2],[5]. That the syntax of the language is very similar to the previous languages and in contraction.
\par
It is said that you can create KML files with the Google Earth user interface, or you can use an XML or simple text editor to enter "raw" KML from scratch[3]. Thus the use is highly user friendly.
\par
Further more, just as web browsers display HTML files, Earth browsers such as Google Earth display KML files[3],[4].
\par
To share your KML and KMZ files, you can e-mail them, host them locally for sharing within a private internet, or host them publicly on a web server[3].
\par
Another site also affirms that KML allows you to draw points, lines, and polygons on maps and globes and share them with others.[4]
\par
KML is an XML grammar and file format, tag names are case-sensitive[5]. It must appear exactly as the syntax is.
\noident
\noident
\noident
\noident
\noident
\noident
\noident
\noident
\noident
\noident
\noident

\noindent \textbf{References}

[1] Wikipedia, \emph{Keyhole Markup Language} at \url{https://en.m.wikipedia.org/wiki/eyhole_Markup_Language.}
[2] Frank Taylor, \emph{Google Earth -Google earth Files KML/KMZ September 5, 2005.} \url{https://www.gearthblog.com/blog/archives/2005/09/google_earth_fi.html}
[3] KML Documentation Introduction \emph{Creating and sharing KML files last updated February 9, 2016.} \url{https:https://doc.arcgis.com/en/arcgis-online/reference/kml.htm//developers.google.com}
[4] ArcGIS Online help. \url{https://doc.arcgis.com/en/arcgis-online/reference/kml.htm}
[5] KML Reference. \url{https://developers.google.com/kml/documentation/kmlreference} 

\end{document}.

